\documentclass{standalone}
\usepackage{standalone}

\begin{document}
\chapter{Introduction}

Speech is the vocalized form of communication used by humans and some animals, which is based upon the syntactic combination of items drawn from the lexicon \cite{rahman2003continuous}. Speech is the easiest and accessible communication medium. It is the most substantial, homely, adaptable methodology.
\\
As our everyday life is winding up noticeably more counting on technology \cite{karim2002recognition}, Speech recognition is a need to interact effortlessly with machines. We know communication between human and computer are limited. Human needs particular skills to use computers. If the human can use their natural language to communicate with machines then it will be so much easier to anyone to use computers in every sphere of life like searching any information in Google or any other search engine. Although there are some speech recognition research on Bangla like \cite{mohamed2012acoustic, islam2009research, hasnat2007isolated}. We have used a new Bangladeshi Bangla standard  model of Bengali intonation in Kaldi for Bangla speech recognition \cite{khan2016intonation}.

\section{About Speech Recognition}
Recognizing speech is a technique of translating a speech signal of a voice expression into
relating document interpretation, for example, words, phonemes or other dialect items. And Automatic Speech Recognition(ASR) system \cite{huang2014historical, sinha2014speech} converts speech to text so that the system is able to extract semantic meaning from the text. Also, the machine can search automatically using the extracted text from voice. So, nowadays ASR has become an essential researches area. However, applying stochastic Language Model(LM) \cite{kuhn1990cache} alongside Hidden Markov Model(HMM) \cite{mohamed2012acoustic} ASR is getting remarkable performance in current years considering the wake of attempting in Artificial Neural Network(ANN) \cite{dahl2012context}. There is Bangla speech recognition system using ANN based approaches and predictive neural networks for continuous neural network \cite{paul2009bangla, zahner1995artificial, tebelskis1991continuous}

\par Now with the utilization of open research apparatuses like Kaldi, Sphinx, CMU LM, and HTK, it becomes easier to assemble a working ASR system. Among the apparatus, we are determined to use the Kaldi toolkit \cite{povey2011kaldi} as its speech recognizer able to generate high-quality lattices, sufficiently fast and provide FST based framework.

\section{Motivation} 
A very sad news for the Bengali people that though Bangla is the 7th broadly spoken language \cite{sinha2014speech} in the world, there is not enough work done in ASR for Bangla. Moreover, the beginning of the ASR was in around 1930 but for Bangla, the research work has begun around 2000 \cite{huang2014historical}. Besides, there is no large corpus set available for Bangla language that is one of the main concern of us to build a large isolated word Corpus set for Bangla Language. Another important thing is, Pipilika is a Bangla search engine that has no voice search option. Since Pipilika is a product of our university, it inspired us to do something for Pipilika.

\section{Report Structure}
The structure of the report is as follows.\\
Chapter:2-
here we have discussed literature review, related works, and some other previous studies, we have done.
\\
Chapter:3- 
Its all about data collection and preprocessing, leveling.
\\
Chapter:4- 
We have defined our works, procedure and a way to build an ASR.
\\
Chapter:5-
Results of previous works and new work are shown here.
\\
Chapter:6-
We faced some issues while developing the system. We have discussed those.
\\
Chapter:7-
There are so many new works to be done. Those are included here.
\\
Chapter:8-
We have discussed what we have done so far.
\\
\end{document}
